% possible options include font size ('10pt', '11pt' and '12pt'), 
% paper size ('a4paper', 'letterpaper', 'a5paper', 'legalpaper', 'executivepaper' and 'landscape')
% and font family ('sans' and 'roman')
\documentclass[11pt,a4paper,sans]{moderncv}

% style options are 'casual' (default), 'classic', 'banking', 'oldstyle' and 'fancy'
\moderncvstyle{classic}
% color options 'black', 'blue' (default), 'burgundy', 'green', 'grey', 'orange', 'purple' and 'red'
\moderncvcolor{blue}
% to set the default font; use '\sfdefault' for the default sans serif font, '\rmdefault' for the default roman one, or any tex font name
% \renewcommand{\familydefault}{\sfdefault}
% uncomment to suppress automatic page numbering for CVs longer than one page
% \nopagenumbers{}

% adjust the page margins
\usepackage[scale=0.75]{geometry}
% depending on the amount of information in the footer, you need to change this value.
% comment this line out and set it to the size given in the warning
\setlength{\footskip}{136.00005pt}
% if you want to change the width of the column with the dates
% \setlength{\hintscolumnwidth}{3cm}
% for the 'classic' style, if you want to force the width allocated to your name and avoid line breaks. be careful though, the length is normally calculated to avoid any overlap with your personal info; use this at your own typographical risks...
% \setlength{\makecvheadnamewidth}{10cm}

% font loading
% for luatex and xetex, do not use inputenc and fontenc
% see https://tex.stackexchange.com/a/496643
\ifxetexorluatex
  \usepackage{fontspec}
  \usepackage{unicode-math}
  \defaultfontfeatures{Ligatures=TeX}
  \setmainfont{Latin Modern Roman}
  \setsansfont{Latin Modern Sans}
  \setmonofont{Latin Modern Mono}
  \setmathfont{Latin Modern Math} 
\else
  \usepackage[T1]{fontenc}
  \usepackage{lmodern}
\fi

% document language
% FIXME: using spanish breaks moderncv
\usepackage[english]{babel}

% personal data
\name{John}{Doe}
\title{Curriculum Vitæ}
\born{4 July 1776}
% the "postcode city" and "country" arguments can be omitted or provided empty
\address{street and number}{postcode city}{country}
% the optional "type" of the phone can be "mobile" (default), "fixed" or "fax"
\phone[mobile]{+1~(234)~567~890}
\phone[fixed]{+2~(345)~678~901}
\phone[fax]{+3~(456)~789~012}
\email{john@doe.org}
\homepage{www.johndoe.com}

% Social icons
\social[linkedin]{john.doe}
\social[xing]{john\_doe}
\social[twitter]{ji\_doe}
\social[github]{jdoe}
\social[gitlab]{jdoe}
% \social[stackoverflow]{0000000/johndoe}
% \social[bitbucket]{jdoe}
% \social[skype]{jdoe}
% \social[orcid]{0000-0000-000-000}
% \social[researchgate]{jdoe}
% \social[researcherid]{jdoe}
% \social[telegram]{jdoe}
% \social[whatsapp]{12345678901}
% \social[signal]{12345678901}
% \social[matrix]{@johndoe:matrix.org}
% \social[googlescholar]{googlescholarid}


\extrainfo{additional information}
% '64pt' is the height the picture must be resized to, 0.4pt is the thickness of the frame around it 
% (put it to 0pt for no frame) and 'picture' is the name of the image file
\photo[64pt][0.4pt]{picture}
\quote{Some quote}

% bibliography adjustments (only useful if you make citations in your resume, or print a list of publications using BibTeX)
% to show numerical labels in the bibliography (default is to show no labels)
% \makeatletter\renewcommand*{\bibliographyitemlabel}{\@biblabel{\arabic{enumiv}}}\makeatother
\renewcommand*{\bibliographyitemlabel}{[\arabic{enumiv}]}
% to redefine the bibliography heading string ("Publications")
% \renewcommand{\refname}{Articles}

% bibliography with mutiple entries
% \usepackage{multibib}
% \newcites{book,misc}{{Books},{Others}}


\begin{document}

% to typeset your resume in Chinese using CJK
%\begin{CJK*}{UTF8}{gbsn}

%-----       letter       ---------------------------------------------------------
% recipient data
\recipient{Karyn Tan}{Oho Group Ltd\\London Area}
\date{June 26, 2024}
\opening{Dear Madam,}
\closing{Yours faithfully,}
\enclosure[Attached]{curriculum vit\ae{}}          % use an optional argument to use a string other than "Enclosure", or redefine \enclname
\makelettertitle

I am writing to apply for the Graduate Software Engineer position at your FinTech scale-up. As a 2024 graduate with a background in Artificial Intelligence, I am eager to start my career in a dynamic and innovative environment like yours.

I am currently finishing my MSc in Artificial Intelligence at the University of Surrey, with graduation expected in September 2024. My experience as a Part-Time Research Assistant at the University of Cambridge involved working on data pipelines, data cleaning, and machine learning model deployment using Python and SQL. These experiences have helped me develop strong problem-solving skills and technical expertise.

In my previous role as a Senior Software Engineer at Tiatech Health Technologies, I developed Python microservices for DICOM imaging and teleconferencing. This role required efficient time management and the ability to work well in a team.

I am excited about the opportunity to join your team and contribute to developing new data pipelines and deploying features. The mentorship and training you offer, along with the chance to work in a fast-growing and social environment, make this role a perfect fit for me.

Thank you for considering my application. I look forward to discussing how I can contribute to your team.

\makeletterclosing


% CV begins here...
\makecvtitle

\section{Education}
% arguments 3 to 6 can be left empty
\cventry{year--year}{Degree}{Institution}{City}{\textit{Grade}}{Description}
\cventry{year--year}{Degree}{Institution}{City}{\textit{Grade}}{Description}

\section{Master thesis}
\cvitem{title}{\emph{Title}}
\cvitem{supervisors}{Supervisors}
\cvitem{description}{Short thesis abstract}

\section{Experience}
\subsection{Vocational}
\cventry{year--year}{Job title}{Employer}{City}{}{General description no longer than 1--2 lines.\newline{}
Detailed achievements:
\begin{itemize}
\item Achievement 1
\item Achievement 2 (with sub-achievements)
\begin{itemize}
  \item Sub-achievement (a);
  \item Sub-achievement (b), with sub-sub-achievements (don't do this!);
\begin{itemize}
  \item Sub-sub-achievement i;
  \item Sub-sub-achievement ii;
  \item Sub-sub-achievement iii;
\end{itemize}
  \item Sub-achievement (c);
\end{itemize}
\item Achievement 3
\item Achievement 4
\end{itemize}}
\cventry{year--year}{Job title}{Employer}{City}{}{Description line 1\newline{}Description line 2\newline{}Description line 3}
\subsection{Miscellaneous}
\cventry{year--year}{Job title}{Employer}{City}{}{Description}

\section{Languages}
\cvitemwithcomment{Language 1}{Skill level}{Comment}
\cvitemwithcomment{Language 2}{Skill level}{Comment}
\cvitemwithcomment{Language 3}{Skill level}{Comment}
\cvitemwithcomment{Language 4}{Skill level}{Comment}

\section{Computer skills}
\cvdoubleitem{category 1}{XXX, YYY, ZZZ}{category 4}{XXX, YYY, ZZZ}
\cvdoubleitem{category 2}{XXX, YYY, ZZZ}{category 5}{XXX, YYY, ZZZ}
\cvdoubleitem{category 3}{XXX, YYY, ZZZ}{category 6}{XXX, YYY, ZZZ}

\section{Skill matrix}
\cvitem{Skill matrix}{Alternatively, provide a skill matrix to show off your skills}
%% Skill matrix as an alternative to rate one's skills, computer or other. 

%% Adjusts width of skill matrix columns. 
%% Usage \setcvskillcolumns[<width>][<factor>][<exp_width>]
%% <width>, <exp_width> should be lengths smaller than \textwidth, <factor> needs to be between 0 and 1.
%% Examples:
% \setcvskillcolumns[5em][][]%    adjust first column. Same as \setcvskillcolumns[5em]
% \setcvskillcolumns[][0.45][]%   adjust third (skill) column. Same as \setcvskillcolumns[][0.45]
% \setcvskillcolumns[][][\widthof{``Year''}]%     adjust fourth (years) column.
% \setcvskillcolumns[][0.45][\widthof{``Year''}]%
% \setcvskillcolumns[\widthof{``Languag''}][0.48][]
% \setcvskillcolumns[\widthof{``Languag''}]%

%% Adjusts width of legend columns. Usage \setcvskilllegendcolumns[<width>][<factor>]
%% <factor> needs to be between 0 and 1. <width> should be a length smaller than \textwidth
%% Examples:
% \setcvskilllegendcolumns[][0.45]
% \setcvskilllegendcolumns[\widthof{``Legend''}][0.45]
% \setcvskilllegendcolumns[0ex][0.46]% this is usefull for the banking style

%% Add a legend if you are using \cvskill{<1-5>} command or \cvskillentry
%% Usage \cvskilllegend[*][<post_padding>][<first_level>][<second_level>][<third_level>][<fourth_level>][<fifth_level>]{<name>}
% \cvskilllegend % insert default legend without lines
\cvskilllegend*[1em]{}% adjust post spacing
% \cvskilllegend*{Legend}%  Alternatively add a description string
%% adjust the legend entries for other languages, here German
% \cvskilllegend[0.2em][Grundkenntnisse][Grundkenntnisse und eigene Erfahrung in Projekten][Umfangreiche Erfahrung in Projekten][Vertiefte Expertenkenntnisse][Experte\,/\,Spezialist]{Legende}

%% Alternative legend style with the first three skill levels in one column
%% Usage \cvskillplainlegend[*][<post_padding>][<first_level>][<second_level>][<third_level>][<fourth_level>][<fifth_level>]{<name>}
% \setcvskilllegendcolumns[][0.6]%  works for classic, casual, banking
% \setcvskilllegendcolumns[][0.55]%  works better for oldstyle and fancy
% \cvskillplainlegend{}
% \cvskillplainlegend[0.2em][Grundkenntnisse][Grundkenntnisse und eigene Erfahrung in Projekten][Umfangreiche Erfahrung in Projekten][Vertiefte Expertenkenntnisse][Experte/Guru]{Legende}

%% Add a head of the skill matrix table with descriptions.
%% Usage \cvskillhead[<post_padding>][<Level>][<Skill>][<Years>][<Comment>]%
% this inserts the standard legend in english and adjust padding
\cvskillhead[-0.1em]
%% Adjust head of the skill matrix for other languages
% \cvskillhead[0.25em][Level][F\"ahigkeit][Jahre][Bemerkung]

%% \cvskillentry[*][<post_padding>]{<skill_cathegory>}{<0-5>}{<skill_name>}{<years_of_experience>}{<comment>}% 
%% Example usages:
\cvskillentry*{Language:}{3}{Python}{2}{I'm so experienced in Python and have realised a million projects. At least.}
\cvskillentry{}{2}{Lilypond}{14}{So much sheet music! Man, I'm the best!}
\cvskillentry{}{3}{\LaTeX}{14}{Clearly I rock at \LaTeX}
% notice the use of the starred command and the optional
\cvskillentry*{OS:}{3}{Linux}{2}{I only use Archlinux btw}
\cvskillentry*[1em]{Methods}{4}{SCRUM}{8}{SCRUM master for 5 years}
%% \cvskill{<0-5>} command
% \cvitem{\textbackslash{cvskill}:}{Skills can be visually expressed by the \textbackslash{cvskill} command, e.g. \cvskill{2}}

\section{Interests}
\cvitem{hobby 1}{Description}
\cvitem{hobby 2}{Description}
\cvitem{hobby 3}{Description}

\section{Extra 1}
\cvlistitem{Item 1}
\cvlistitem{Item 2}
\cvlistitem{Item 3. This item is particularly long and therefore normally spans over several lines. Did you notice the indentation when the line wraps?}

\section{Extra 2}
\cvlistdoubleitem{Item 1}{Item 4}
\cvlistdoubleitem{Item 2}{Item 5}
\cvlistdoubleitem{Item 3}{Item 6. Like item 3 in the single column list before, this item is particularly long to wrap over several lines.}

\section{References}
\begin{cvcolumns}
  \cvcolumn{Category 1}{\begin{itemize}\item Person 1\item Person 2\item Person 3\end{itemize}}
  \cvcolumn{Category 2}{Amongst others:\begin{itemize}\item Person 1, and\item Person 2\end{itemize}(more upon request)}
  \cvcolumn[0.5]{All the rest \& some more}{\textit{That} person, and \textbf{those} also (all available upon request).}
\end{cvcolumns}

% Publications from a BibTeX file without multibib
%  for numerical labels: \renewcommand{\bibliographyitemlabel}{\@biblabel{\arabic{enumiv}}}% CONSIDER MERGING WITH PREAMBLE PART
%  to redefine the heading string ("Publications"): \renewcommand{\refname}{Articles}
% \nocite{*}
% \bibliographystyle{plain}
% 'publications' is the name of a BibTeX file
% \bibliography{publications}

% Publications from a BibTeX file using the multibib package
% \section{Publications}
% \nocitebook{book1,book2}
% \bibliographystylebook{plain}
% 'publications' is the name of a BibTeX file
% \bibliographybook{publications}
% \nocitemisc{misc1,misc2,misc3}
% \bibliographystylemisc{plain}
% 'publications' is the name of a BibTeX file
% \bibliographymisc{publications}

\clearpage

% if you are typesetting your resume in Chinese using CJK; the \clearpage is required for fancyhdr to work correctly
% with CJK, though it kills the page numbering by making \lastpage undefined
% \end{CJK*}

\end{document}
